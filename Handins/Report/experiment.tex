%!TEX root = ../main.tex

In order to test the results a similar setup as the one shown in figures 2 and 3 will be used, showing real objects made out of different materials, so at least a matte, a shiny and a reflective object are present. The scene will be augmented with a virtual object with the possibility of changing the shader used for it. At least a Lambert diffuse, a Phong or Blinn and a specular shaders will be available. The setup will have known,  consistent and controllable lighting.\newline
Besides augmenting the scene with the developed application images will be produced manually, replicating the real world lighting as close as possible. Screen captures of the plain real world scene, with no augmentation whatsoever; the manually augmented scene and the automatic results provided by the method will be left for the reader to see. Also criticizing and discussing how much the results obtained match the expected results and the ground truth.\newline
The actual way in which the experiment will be conducted is defined in the following scenario:
\begin{itemize}
    \item \textbf{Setting:} A room with consistent and invariable lighting will be used. A set of common objects will be laid on a table, the objects must include a matte, a shiny and a reflective object. At least one extra light source with known parameters.
    \item \textbf{Requirements:} A mobile device running the developed demo application, a marker for virtual objects. A real object and its corresponding virtual counterpart, modelled as close as possible. 
    \item \textbf{Goals:} Obtaining a set of images that will enable readers and experimenters alike to make a fair comparison of the method application, a commercial solution and a real object.
    \item \textbf{Actions:} Place the real object on top of the marker and take a picture with the same device running the demo application. Then remove the object and take screen captures of the augmented scene using a commercial solution and subsequently the method demo application.
\end{itemize}
