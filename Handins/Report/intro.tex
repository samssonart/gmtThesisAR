%!TEX root = ../main.tex

The rendering of virtual objects on current AR applications has a number of shortcomings in the state of the art. It is quite easy to tell that it’s a superimposed virtual object. There are many factors that come into play and give away the lack of feasibility of an AR object, such as marker jiggle, inaccurate depth cues, lack of occlusion from real life objects and incompatible lighting.
The method proposed in this work addresses the latter problem, the incompatible light conditions of the virtual and real world. The method is based on the idea that a 360 photograph of the environment can give us a lot of information about the lighting conditions and that such 360 photographs are now simple to create on an average mobile device. From this 360 photograph it can be determined if the environment is outdoors by day or any other case of illumination. If the environment is indoors the light sources of the room will be approximated using the photo, if outdoors the sun will be assumed as the main source of light, and its position would be calculated using the clock, calendar and compass of the mobile device running the application. \newline
Whatever the case, the method's output is a set of lights whose properties are a position relative to the camera, a rotation, color, type of light (spotlight or area light) and an intensity.
It's important to stress the fact that the method is proposed for mobile devices. There have been previous works, as detailed in the next section, that deal with the same problem but on computers. Mobile devices present advantages over computers for this task, but also the downside of generally lower processing power compared to computers. This decision comes from the ease of producing the 360 photographs locally, from the judgement that AR is better suited for mobile devices than computers (for mobility and ease of use reasons) and also to be able to exploit the devices geolocation hardware. Of course this doesn't mean that the method will only be applicable to mobile, there are also ways to adapt it to work on computers too.\newline
The research question from which this method was originally envisioned is: "Can the accuracy of lighting in AR applications be improved using a mobile device to it's full potential?". What this means is that thanks to the many ways a smartphone or tablet can interact with the user and the environment we can know a lot about the surroundings. And this information could potentially be very useful to save computation time, which is the main downside to using mobile devices, their relatively lower computational capabilities.\newline
The contributions of this method are:
\begin{itemize}
    \item An image-based method with which the lighting conditions of the full visible environment can be modeled
    \item The use of the mobile device's geolocation hardware to enhance the light simulation with the knowledge of the current weather conditions and the position of the sun
\end{itemize}