\documentclass{article}
\usepackage[utf8]{inputenc}
\usepackage{natbib}
\usepackage{graphicx}

\title{Realtime Environment Lighting Simulation for Augmented Reality  Applications}
\author{Marco Antonio Salcedo Sanson }
\date{February 2017}

% Definition of \maketitle
\makeatletter         
\def\@maketitle{
\noindent
\includegraphics[scale = 0.5]{Figures/uu-logo.png}\\[8ex]
\begin{center}
{\huge \bfseries \sffamily \@title }\\[4ex] 
{\Large  \@author}\\[4ex] 
\@date\\[8ex]
\end{center}}
\makeatother


\begin{document}

\maketitle

\section{Introduction}
The rendering of virtual objects on current AR applications has a number of shortcomings in the state of the art. It is quite easy to tell that it’s a superimposed virtual object. There are many factors that come into play and give away the lack of feasibility of an AR object, such as marker jiggle, inaccurate depth cues, lack of occlusion from real life objects and incompatible lighting.
The method proposed in this work addresses the latter problem, the non-matching light conditions of the virtual and real world. The method is based on the idea that a 360 photograph of the environment can give us a lot of information about the lighting conditions and that such 360 photographs are now simple to create on an average mobile device. From this 360 photograph it can be determined if the environment is outdoors by day or any other case of illumination. If the environment is indoors the light sources of the room will be approximated using the photo, if outdoors the sun will be assumed as the main source of light, and its position would be calculated using the clock, calendar and compass of the mobile device running the application. 
Whatever the case, the method's output is a set of lights whose properties are a position relative to the camera, a rotation and an intensity.
It's important to stress the fact that the method is proposed for mobile devices. There have been previous works, as detailed in the next section, that deal with the same problem but on computers. Mobile devices present advantages over computers for this task, but also the downside of generally lower processing power compared to computers. This decision comes from the ease of producing the 360 photographs locally, from the judgement that AR is better suited for mobile devices than computers (for mobility and ease of use reasons) and also to be able to exploit the devices geolocation hardware. Of course this doesn't mean that the method will only be applicable to mobile, there are also ways to adapt it to work on computers too.
The contributions of this method are an image-based method with which the lighting conditions of the full visible environment can be modeled and the use of the mobile device's geolocation hardware to enhance the light simulation with the knowledge of the current weather conditions and the position of the sun, which also can be used to fall back in case of extremely complex lighting conditions.
\section{Related Work}
Agusanto et al. proposed a similar method \cite{agusanto2003}. The main differences in the method they propose have to do mostly with the technologies used. Being already an old paper, their method was conceived for Augmented Reality on a computer, rather than on a smartphone. My choice of a smartphone comes with the downside of lower processing power, but with the great advantage of the different sensors available on the device. The method in the Agusanto paper infers the lighting conditions using hand-crafted HDR photographs of the real scene using highly reflective metal spheres in contrast with my proposition of using the 360 degree photograph on the device itself, which is a lot more accessible to an average consumer.
Pessoa\cite{pessoa2011} expanded on Agusanto's work and created the Real-Time Photorealistic Rendering of Synthetic Objects into Real Scenes (RPR-SORS) toolkit as an extension of the ARToolkit Augmented reality SDK. Although they use more modern techniques, such as cubemaps instead of HDR photographs, their approach still has the same differences than the Agusanto method.
There have been several methods to insert virtual objects into static photographs automatically with highly realistic results. One such example is the method by Karsh \cite{karsh2014} These kind of methods are not suitable for real- time applications, and they use a single view image to infer the lighting conditions. The method proposed here is more robust in the sense that it uses a 360 view of the environment to eliminate assumptions of what there is out of frame.
Xing\cite{xing2013} also devised a method to compose virtual objects into static photographs, focusing on outdoors scenes. The method is quite sophisticated, but a lot of the information fed to the light calculation functions involve manual input by the user. The proposed method is intended to work both for indoors and outdoors scenes, the main difference with the way outdoors scenes will be managed is that the method works automatically, without user intervention. This is accomplished by the usage of the mobile device's dedicated geolocation hardware. By using it, there is no need to assume anything or ask for any user input, the device will be able to tell a good approximation of the sun position using only the sensor's information.
\section{Theoretical Framework}
In this section a few technical concepts will be explored in order to make the proposed method and what it aims to accomplish more clear.

\textbf{Augmented Reality}\newline
As of yet, the definition of Augmented Reality that is most widely accepted is the one presented by Azuma\cite{azuma1997}. In this survey Augmented Reality (or AR for short) is defined as a software system that features a blend of real and virtual objects, the user can interact with it in real time and presents a 2D representation of a 3D world for both the real and virtual sides. A feature film that integrates real and virtual characters or objects is therefore strictly speaking not AR for example, because it's static content, the consumer cannot interact with it at all, let alone in real time.
In order to combine real and virtual objects both worlds must be aligned, in the sense that they must share a common origin for both position and rotation. If the virtual objects present in the real scene are not aligned the feasibility of the whole composition is compromised. The tracking of real world features as a guideline for alignment in the virtual world is one of the challenges in the field of AR. There are many techniques that have been used, such as sensor-based tracking, that uses dedicated hardware to feed the software application the position and rotation information it needs; vision-based tracking, in which the position and orientation origin of the virtual world are determined by images recognized by a computer vision algorithm. These images are named fiducial markers and can be an artificial black and white image, a "natural" images (points, lines, shapes and/or textures) or even 3D objects. Other approaches involving both sensors and vision algorithms have also been used, if the reader is interested in learning more about tracking, Zhou's survey\cite{zhou2008} on tracking is a good resource.

\textbf{Sensors}\newline
Most smartphones and tablets in the market nowadays have built-in sensors to measure many things that are helpful for application developers, such as motion, orientation and other environmental conditions. The ones that are relevant for this method are the following:\newline
Accelerometer: This device measures proper acceleration relative to gravity. Its application in mobile development is to measure motion changes and to measure the device orientation relative to Earth's surface. It usually consists of 3 orthogonal axes. \newline
Gyroscope: A sensor that is capable of measuring the rate of rotation around a particular axis. It serves the same purpose as the accelerometer, but mobile devices usually have both for more robust measurements. \newline
Photometer: There is a wide variability in terms of capacity of photometers across different devices, but the only measurement that is guaranteed is the ambient illuminance. \newline 
Magnetometer: In terms of mobile devices it serves the purpose of a compass, it measures the device's orientation with respect to the Earth's magnetic poles. In mobile development it is useful to get the heading of the device.\newline
GPS: Measures the raw position of the device in three-dimmensional Cartesian coordinates with origin on the center of the Earth. It can be used to determined where on planet Earth the device is located (Country, city, neigborhood, etc). It needs to have line of sight with no electromagnetic interference with at least 4 out of the 24 satellites in orbit that are used for GPS. \newline
\section{Method}
This section covers the steps needed to produce the wanted results. The input needed is a marker M to provide the object position in the real world; and a 3D mesh O which will be rendered in Augmented Reality. The output is a set of light sources L\textsubscript{i}, each one having a position, orientation and intensity. In order to complete the luminance (luma) analysis a 360 panoramic image is required, but it will be generated as a previous step, it is not necessary to have one beforehand.

1.- Capture a 360 panoramic image of the environment using the device. In order to perform a luminance analisys it's necessary to lose the color information, this is done by desaturating the image. After that the contrast of the image is adjusted to preserve only the areas of high luminance.

2.- Scan for zones of high luminance. This is done by a computer vision blob detection algorithm, since after the image preparation stages the areas of high luminance are just that.

2b.- High luminance zone found: Analyze the contour of the object, if it's solid it's a false positive, probably a clear object in the visual range. If it's diffuse and/or jagged a light source was found. 
2c.- If there is a continuous high luminance area encompassing most of the image's upper section, environment is outdoors, only sunlight emulation will be used. Even if there are significant occlusions causing discontinuity in the high luminance area it can be classified as an outdoors scene if the height of the light sources is at least 30\% of the image height for all instances.

2c.- No high luminance area found: light source is not visible in the picture, will be approximated with further analysis. This is based around the intuition that it's not possible that there is not a light source, unless the image was pitch black.

3.- Light position and orientation calculation: the input 360 image is mapped on a unit cylinder, the preliminary known light sources' positions will be obtained in cylindrical coordinates in this step. They are then converted to Cartesian coordinates and the distance from the camera will be adjusted once the orientation is known. If the light source is not present in frame it will be assumed to be directly above the camera, because the only positions that a 360 degree photograph would miss are directly above or directly below the camera, and a light source located directly below the camera would be highly unlikely in normal situations.


\section{Implementation}
Here goes the implementation and technical details.

\section{Conclusion}
results seem promising but more research is necessary

\bibliographystyle{plain}
\bibliography{references}
\end{document}
